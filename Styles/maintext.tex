\documentclass{article}


% if you need to pass options to natbib, use, e.g.:
%     \PassOptionsToPackage{numbers, compress}{natbib}
% before loading neurips_2024


% ready for submission
\usepackage{neurips_2024}


% to compile a preprint version, e.g., for submission to arXiv, add add the
% [preprint] option:
%     \usepackage[preprint]{neurips_2024}


% to compile a camera-ready version, add the [final] option, e.g.:
%     \usepackage[final]{neurips_2024}


% to avoid loading the natbib package, add option nonatbib:
%    \usepackage[nonatbib]{neurips_2024}


\usepackage[utf8]{inputenc} % allow utf-8 input
\usepackage[T1]{fontenc}    % use 8-bit T1 fonts
\usepackage{hyperref}       % hyperlinks
\usepackage{url}            % simple URL typesetting
\usepackage{booktabs}       % professional-quality tables
\usepackage{amsfonts}       % blackboard math symbols
\usepackage{nicefrac}       % compact symbols for 1/2, etc.
\usepackage{microtype}      % microtypography
\usepackage{xcolor}         % colors

\usepackage{biblatex} %Imports biblatex package
\addbibresource{bibliography.bib} %Import the bibliography file

\title{Layout Design with Large Language Models: Opportunities and Challenges}

\author{%
  Bo Wen\thanks{IBM Watson Research Center, Yorktown Heights, NY, USA} \\
  \texttt{bwen@us.ibm.com} \\
  \And
  Xin Zhang\thanks{IBM Watson Research Center, Yorktown Heights, NY, USA} \\
  \texttt{xzhang@us.ibm.com} \\
}

\begin{document}

\maketitle

\begin{abstract}
  % Update abstract to reflect the content of the paper
  This paper explores the potential of large language models (LLMs) as a "layout design copilot" in various domains such as semiconductor, IC, and microfluidics. We evaluate the capabilities of LLMs in generating basic elements and combining them into complex designs. Our experiments focus on via connections, microfluidics channel design, and fiducial marker generation. We discuss the opportunities and challenges of using LLMs in layout design, highlighting the importance of providing expert knowledge and context to improve their performance.
\end{abstract}

\section{Introduction}
% Brief overview of layout design in various domains
% Challenges in automating repetitive tasks while maintaining flexibility 
% Potential of large language models (LLMs) as a "layout design copilot"
% Thesis statement and paper overview

\section{Methodology}
\subsection{Basic Element Generation}
% Evaluation of popular LLMs in drawing basic shapes (zero-shot attempt)
% Suggestion: Present a table showcasing success rates
% Conclusion: Setting up a RAG system with working example code improves accuracy

\subsection{Combining Basic Elements into Complicated Designs}
% Introduction to the three explored use cases
\paragraph{Via connection in semiconductor process}
\paragraph{Microfluidics channel design}
\paragraph{Fiducial marker generation}

\section{Results and Discussion}
\subsection{Via Connection Experiment}
% Detailed analysis of the via connection experiment (refer to via.pdf)
% Discuss the mistakes made by GPT-4 and their potential causes
% Highlight the importance of providing expert knowledge and context to LLMs

\subsection{Microfluidics Design}
% LLMs can guide researchers in estimating channel size based on fluid properties and dynamics
% Limitations: LLMs struggle with arithmetic calculations and may lack domain expertise

\subsection{Fiducial Marker Generation}
% LLMs perform well in generating fiducial markers on grids or arrays
% Advantages over commercial software: flexibility and potential other benefits

\subsection{General Discussion}
% The importance of spatial thinking, imagination, and understanding of physics for LLMs
% Need for providing use case requirements and expert knowledge as context to LLMs

\section{Conclusion and Future Work}
% Recap the main findings and insights
% Discuss the potential of LLMs in layout design and the challenges to overcome
% Outline future research directions:
%   - Dealing with more complicated designs
%   - Creating a benchmark to test LLM capabilities in layout design tasks
% Final thoughts on the role of LLMs as a "layout design copilot"

\printbibliography %Prints bibliography

\end{document}